% Options for packages loaded elsewhere
\PassOptionsToPackage{unicode}{hyperref}
\PassOptionsToPackage{hyphens}{url}
%
\documentclass[
]{article}
\usepackage{amsmath,amssymb}
\usepackage{lmodern}
\usepackage{ifxetex,ifluatex}
\ifnum 0\ifxetex 1\fi\ifluatex 1\fi=0 % if pdftex
  \usepackage[T1]{fontenc}
  \usepackage[utf8]{inputenc}
  \usepackage{textcomp} % provide euro and other symbols
\else % if luatex or xetex
  \usepackage{unicode-math}
  \defaultfontfeatures{Scale=MatchLowercase}
  \defaultfontfeatures[\rmfamily]{Ligatures=TeX,Scale=1}
\fi
% Use upquote if available, for straight quotes in verbatim environments
\IfFileExists{upquote.sty}{\usepackage{upquote}}{}
\IfFileExists{microtype.sty}{% use microtype if available
  \usepackage[]{microtype}
  \UseMicrotypeSet[protrusion]{basicmath} % disable protrusion for tt fonts
}{}
\makeatletter
\@ifundefined{KOMAClassName}{% if non-KOMA class
  \IfFileExists{parskip.sty}{%
    \usepackage{parskip}
  }{% else
    \setlength{\parindent}{0pt}
    \setlength{\parskip}{6pt plus 2pt minus 1pt}}
}{% if KOMA class
  \KOMAoptions{parskip=half}}
\makeatother
\usepackage{xcolor}
\IfFileExists{xurl.sty}{\usepackage{xurl}}{} % add URL line breaks if available
\IfFileExists{bookmark.sty}{\usepackage{bookmark}}{\usepackage{hyperref}}
\hypersetup{
  pdftitle={Home},
  hidelinks,
  pdfcreator={LaTeX via pandoc}}
\urlstyle{same} % disable monospaced font for URLs
\usepackage[margin=1in]{geometry}
\usepackage{longtable,booktabs,array}
\usepackage{calc} % for calculating minipage widths
% Correct order of tables after \paragraph or \subparagraph
\usepackage{etoolbox}
\makeatletter
\patchcmd\longtable{\par}{\if@noskipsec\mbox{}\fi\par}{}{}
\makeatother
% Allow footnotes in longtable head/foot
\IfFileExists{footnotehyper.sty}{\usepackage{footnotehyper}}{\usepackage{footnote}}
\makesavenoteenv{longtable}
\usepackage{graphicx}
\makeatletter
\def\maxwidth{\ifdim\Gin@nat@width>\linewidth\linewidth\else\Gin@nat@width\fi}
\def\maxheight{\ifdim\Gin@nat@height>\textheight\textheight\else\Gin@nat@height\fi}
\makeatother
% Scale images if necessary, so that they will not overflow the page
% margins by default, and it is still possible to overwrite the defaults
% using explicit options in \includegraphics[width, height, ...]{}
\setkeys{Gin}{width=\maxwidth,height=\maxheight,keepaspectratio}
% Set default figure placement to htbp
\makeatletter
\def\fps@figure{htbp}
\makeatother
\setlength{\emergencystretch}{3em} % prevent overfull lines
\providecommand{\tightlist}{%
  \setlength{\itemsep}{0pt}\setlength{\parskip}{0pt}}
\setcounter{secnumdepth}{-\maxdimen} % remove section numbering
\ifluatex
  \usepackage{selnolig}  % disable illegal ligatures
\fi

\title{Home}
\author{}
\date{\vspace{-2.5em}}

\begin{document}
\maketitle

\hypertarget{sky-onosson-ph.d.}{%
\section[Sky Onosson, Ph.D.~]{\texorpdfstring{Sky Onosson,
Ph.D.~\href{mailto:sky@onosson.com}{\protect\includegraphics[width=1em,height=1em]{index_files/figure-latex/fa-icon-5777426a4891959e8c9a098fe5069ff6.pdf}}}{Sky Onosson, Ph.D.~}}\label{sky-onosson-ph.d.}}

This site contains information on my research \& work as a linguist. I
am currently a postdoctoral researcher, some-times sessional lecturer,
and active job applicant open to interesting opportunities of all kinds.

If you'd like to get in touch with me, the best way is by email at
\href{mailto:sky@onosson.com}{\nolinkurl{sky@onosson.com}}.

\textbf{Institutional Address and Email} Department of Linguistics
University of Manitoba 534 Fletcher Argue Building 15 Chancellor's
Circle Winnipeg MB Canada R3T 5V5

\href{mailto:Sky.Onosson@umanitoba.ca}{\nolinkurl{Sky.Onosson@umanitoba.ca}}

ORCID: \href{https://orcid.org/0000-0001-9308-7202}{0000-0001-9308-7202}

\begin{center}\rule{0.5\linewidth}{0.5pt}\end{center}

\hypertarget{on-this-page}{%
\subsubsection{On this page}\label{on-this-page}}

\begin{itemize}
\tightlist
\item
  \protect\hyperlink{work}{My work in linguistics}
\item
  \protect\hyperlink{projects}{My research projects}
\item
  \protect\hyperlink{cv}{My CV}
\end{itemize}

\begin{center}\rule{0.5\linewidth}{0.5pt}\end{center}

\hypertarget{work}{%
\subsection[My work in linguistics ]{\texorpdfstring{My work in
linguistics
\href{pdf/OnossonCV.pdf}{\protect\includegraphics[width=0.75em,height=1em]{index_files/figure-latex/fa-icon-e71e84c2bd77dbcf0cb8f98a3ea09d57.pdf}}}{My work in linguistics }}\label{work}}

My research revolves around the fields of phonetics, phonology,
sociolinguistics, and computational linguistics. A lot of my work has
looked at the temporal/dynamic properties of vowel production, such as
diphthongs, from both phonological and phonetic points of view.

I am currently working as a Postdoctoral Research Fellow in the
\href{https://umanitoba.ca/linguistics/}{Department of Linguistics} at
the \href{https://www.umanitoba.ca/}{University of Manitoba} as part of
the
\href{https://home.cc.umanitoba.ca/~rosenn/sace.html}{\emph{Languages in
the Prairies Project}} spearheaded by
\href{https://home.cc.umanitoba.ca/~rosenn/}{Dr.~Nicole Rosen} (see
below).

I am a graduate of both the
\href{http://www.umanitoba.ca/linguistics/}{University of Manitoba}
(B.A. and M.A.) and the
\href{https://www.uvic.ca/humanities/linguistics/}{University of
Victoria} (Ph.D.).

My academic work and research experience includes teaching courses in
Linguistics at the
\href{http://www.umanitoba.ca/linguistics/}{University of Manitoba} and
the
\href{https://www.uwinnipeg.ca/interdisciplinary-linguistics/index.html}{University
of Winnipeg}, serving as a research assistant and lab instructor at the
\href{https://www.uvic.ca/humanities/linguistics/}{University of
Victoria}, presenting at a variety of major linguistics conferences, and
volunteering as editor and conference organizer on multiple occasions.

\href{pdf/OnossonCV.pdf}{My academic CV} contains much more information
on all of the above. I also have an
\href{pdf/OnossonResume.pdf}{industry-facing resume} for non-academic
purposes, should they arise.

\textbf{Other links:}

\begin{itemize}
\tightlist
\item
  \href{https://github.com/onosson}{\includegraphics[width=0.97em,height=1em]{index_files/figure-latex/fa-icon-306b3f680723aa2b0a39da2cc59a622e.pdf}
  GitHub}
\item
  \href{https://www.researchgate.net/profile/Sky_Onosson}{\includegraphics[width=0.88em,height=1em]{index_files/figure-latex/fa-icon-07ac794574a6b64fa79ea2d17cf66cb7.pdf}
  ResearchGate}
\item
  \href{https://www.linkedin.com/in/sky-onosson-57902870/}{\includegraphics[width=0.88em,height=1em]{index_files/figure-latex/fa-icon-c55c9b7eac4bc6a66901abe57b2a6c7d.pdf}
  LinkedIn}
\item
  \href{https://twitter.com/onosson}{\includegraphics[width=1em,height=1em]{index_files/figure-latex/fa-icon-63ba3f107f112695af1559dbf0eb8614.pdf}
  Twitter}
\end{itemize}

\begin{center}\rule{0.5\linewidth}{0.5pt}\end{center}

\hypertarget{projects}{%
\subsection[Research projects ]{\texorpdfstring{Research projects
\href{pdf/OnossonCV.pdf}{\protect\includegraphics[width=0.75em,height=1em]{index_files/figure-latex/fa-icon-e71e84c2bd77dbcf0cb8f98a3ea09d57.pdf}}}{Research projects }}\label{projects}}

I am currently involved in several active research projects which
include, in no particular order:

\begin{itemize}
\item
  \textbf{Vowel dynamics.} This work stems from my
  \href{pdf/Onosson\%20-\%202010\%20-\%20MA\%20Thesis.pdf}{M.A.} and
  \href{pdf/Onosson\%20-\%202018\%20-\%20PhD\%20Dissertation.pdf}{Ph.D.}
  graduate work, which focused on ``Canadian Raising'' of the /aj/ and
  /aw/ diphthongs before voiceless codas. I am interested in the ways
  that different dialects handle pre-voiceless shortening
  phonologically, and how this plays out phonetically in the context of
  what are perceived as ``raised'' diphthongal nuclei. I have also been
  expanding my work on diphthongs and diphthongal productions into
  non-IndoEuropean languages such as \emph{Media Lengua} (Ecuador) and
  \emph{Hul'q'umi'num'} (British Columbia) -- see below.
\item
  \textbf{Ethnoloinguistic variation in \emph{Manitoba/Prairie
  English}.} The data for this research is drawn from
  \href{https://home.cc.umanitoba.ca/~rosenn/}{Dr.~Nicole Rosen's}
  Canada Research Chair-funded
  \href{https://home.cc.umanitoba.ca/~rosenn/sace.html}{\emph{Languages
  in the Prairies Project}} (LIPP). Significant findings from this
  research include the documentation of distinctive ethnolinguistic
  patterns across several phonological processes in the Filipino and
  Mennonite communities in Manitoba, and documentation of both Canadian
  Shift and prevelar raising in general Manitoba/Prairie English. My
  research on LIPP has led to two
  \href{pdf/Onosson,\%20Rosen\%20-\%202020\%20-\%20American\%20Dialect\%20Society.pdf}{conference
  presentations}, including a
  \href{pdf/Onosson,\%20Rosen,\%20Li\%20-\%202019\%20-\%20ICPhS\%20Proceedings.pdf}{published
  proceedings}, and a full article is currently under peer review.
\item
  \textbf{Variation and change in contemporary \emph{Brazilian
  Portuguese}.} Beginning in 2015, in collaboration with
  \href{https://artsandscience.usask.ca/linguistics/graduates/current-graduate-students.php}{Christiani
  Pinheiro Thompson} (University of Saskatchewan), we have recorded more
  than 80 multi-media intervews with nearly 180 middle-school children
  in urban Rio de Janeiro. Currently we are focusing on carrying out the
  lengthy process of transcription, after which we plan to document many
  of the
  \href{pdf/Thompson,\%20Onosson\%20-\%202016\%20-\%20New\%20Ways\%20of\%20Analyzing\%20Variation\%2045.pdf}{interesting
  linguistic innovations} taking place in this community, which include
  lexical, syntactic, and phonological features among others. I
  currently have two articles in prep related to this project.
\item
  \textbf{The vowels of \emph{Media Lengua}.} This project is a
  collaboration with \href{http://www.jessestewart.net/}{Dr.~Jesse
  Stewart} (University of Saskatchewan). Media Lengua is a mixed
  language spoken in Ecuador whose lexicon is mainly Spanish-origin,
  fitted into a vowel system which more closely resembles that of
  Quichua (and which utilizes mainly Quichua syntax). Our current work
  focuses on how vowel-sequences from different-origin source languages
  are handled in Media Lengua, from both phonetic and phonological
  perspectives. Two papers resulting from this collaboration are
  currently under review, and we have plans to develop more research
  focused on the unique vowel system of Media Lengua.
\item
  \textbf{Phonetic analysis of Indigenous Languages of British
  Columbia.} This work, in which I am very much a junior partner at this
  stage, is being conducted in collaboration with
  \href{https://www.uvic.ca/humanities/linguistics/people/faculty/birdsonya.php}{Dr.~Sonya
  Bird} (University of Victoria), and involves the Coast Salish language
  \emph{Hul'q'umi'num'} spoken on Vancouver Island, and the Athabaskan
  language \emph{Tsilhqot'in} spoken in the interior of B.C. My work on
  Hul'q'umi'num' has involved application of techniques for the analysis
  of vowel dynamics to compare production differences between fluent
  elders and language learners, for the purpose of assisting ongoing
  language revitalization efforts. This study has produced two
  \href{pdf/Onosson\%20-\%202019\%20-\%20Prairie\%20Workshop\%20on\%20Languages\%20and\%20Linguistics\%20V.pdf}{conference
  presentations}, including a
  \href{pdf/Onosson,\%20Bird\%20-\%202019\%20-\%20ICPhS\%20Proceedings.pdf}{published
  proceedings}. My work on Tsilhqot'in is focused on a descriptive
  analysis of consonant sounds; a paper resulting from this study is
  currently under review.
\end{itemize}

\begin{center}\rule{0.5\linewidth}{0.5pt}\end{center}

\hypertarget{cv}{%
\subsection[Curriculum vitae and publication downloads
]{\texorpdfstring{Curriculum vitae and publication downloads
\href{pdf/OnossonCV.pdf}{\protect\includegraphics[width=0.75em,height=1em]{index_files/figure-latex/fa-icon-e71e84c2bd77dbcf0cb8f98a3ea09d57.pdf}}}{Curriculum vitae and publication downloads }}\label{cv}}

This site hosts archived copies of most of my published or presented
work going back to my 2010 master's thesis, all of which are listed on
\href{pdf/OnossonCV.pdf}{my full CV}. The simplified CV presented below
combines peer-reviewed publications, current papers I have in some state
of preparation, conference presentations, invited talks, workshops, and
media interviews, in chronological order, and includes active links to
copies of publications which are available for download:

\begin{longtable}[]{@{}
  >{\raggedright\arraybackslash}p{(\columnwidth - 4\tabcolsep) * \real{0.14}}
  >{\raggedright\arraybackslash}p{(\columnwidth - 4\tabcolsep) * \real{0.43}}
  >{\raggedright\arraybackslash}p{(\columnwidth - 4\tabcolsep) * \real{0.43}}@{}}
\toprule
Year & Title & Journal/Forum/Venue \\
\midrule
\endhead
2021 &
\href{pdf/Onosson,\%20Stewart\%20-\%202021\%20-\%20Language\%20and\%20Speech.pdf}{The
effects of language contact on non-native vowel sequences in lexical
borrowings: The case of Media Lengua} (with Jesse Stewart) & Language
and Speech \\
~ & A multi-method approach to correlate identification in acoustic
data: The case of Media Lengua (with Jesse Stewart) & Laboratory
Phonology (accepted) \\
~ & Phonetic change in the grammaticalization of Brazilian Portuguese
\emph{tipo} (with Christiani Thompson) & Canadian Linguistic
Association \\
~ & A phonetic investigation of Tŝilhqot'in /z/ and
/z\textsuperscript{ʕ}/ (with Sonya Bird) & Under revision \\
~ & Prevelar front-vowel merger in Winnipeg English & Under review \\
~ & Canadian Shift amidst demographic shift: Ethnolinguistic effects in
Western Canadian English vowels (with Nicole Rosen and Lanlan Li) & In
preparation \\
~ & Topicalization in Brazilian Portuguese: Subject-doubling
constructions (with Christiani Thompson) & In preparation \\
2020 &
\href{pdf/Onosson,\%20Stewart\%20-\%202020\%20-\%20Society\%20for\%20Pidgin\%20and\%20Creole\%20Linguistics.pdf}{The
effects of language contact on non-native diphthongs in lexical
borrowings: The case of Media Lengua} (with Jesse Stewart) & Society for
Pidgin and Creole Linguistics, New Orleans LA \\
~ &
\href{pdf/Onosson,\%20Rosen\%20-\%202020\%20-\%20American\%20Dialect\%20Society.pdf}{Ethnolinguistic
Vowel Differentiation in Manitoba English} & American Dialect Society,
New Orleans LA \\
~ & TRAP-raising and the Canadian/Low-Back-Merger Shift in Prairie
English (accepted--did not present due to Covid19) & Canadian Linguistic
Association, Western University \\
2019 & \href{https://github.com/onosson/UM_Workshop}{Workshop on
Statistics for Linguistics} & University of Manitoba \\
~ &
\href{pdf/Onosson,\%20Bird\%20-\%202019\%20-\%20ICPhS\%20Proceedings.pdf}{Differences
In Vowel-glide Production Between L1 And L2 Speakers Of Hul'q'umi'num'}
(with Sonya Bird) & International Congress of Phonetic Sciences,
Melbourne \\
~ &
\href{pdf/Onosson,\%20Rosen,\%20Li\%20-\%202019\%20-\%20ICPhS\%20Proceedings.pdf}{Ethnolinguistic
Differentiation and the Canadian Shift} (with Nicole Rosen and Lanlan
Li) & International Congress of Phonetic Sciences, Melbourne \\
~ &
\href{pdf/Onosson\%20-\%202019\%20-\%20Prairie\%20Workshop\%20on\%20Languages\%20and\%20Linguistics\%20V.pdf}{Acoustic
Phonetics and Language Revitalization in the Hul'q'umi'num' Community}
(with Sonya Bird) & Prairie Workshop on Languages and Linguistics V,
University of Winnipeg \\
2018 &
\href{pdf/Onosson\%20-\%202018\%20-\%20PhD\%20Dissertation.pdf}{An
Acoustic Study of Canadian Raising in Three Dialects of North American
English} & Doctoral dissertation, the University of Victoria \\
~ &
\href{pdf/Roeder,\%20Onosson,\%20D'Arcy\%20-\%202018\%20-\%20Journal\%20of\%20English\%20Linguistics.pdf}{Joining
the Western Region: Sociophonetic Shift in Victoria} (with Rebecca
Roeder and Alex D'Arcy) & Journal of English Linguistics \\
~ &
\href{pdf/Roeder,\%20Onosson\%20-\%202018\%20-\%20Sociolinguistics\%20Symposium\%2022.pdf}{Best
practices in automatic vowel production analysis} & Sociolinguistics
Symposium 22, University of Auckland \\
2017 &
\href{pdf/Onosson\%20-\%202017\%20-\%20Chicago\%20Workshop\%20on\%20Dynamic\%20Modeling\%20in\%20Phonetics\%20and\%20Phonology.pdf}{Canadian
Raising or Canadian Shortening? Comparing and contrasting dynamic models
of vowel duration} & Workshop on Dynamic Modeling in Phonetics \&
Phonology, University of Chicago \\
2016 &
\href{pdf/Onosson\%20-\%202016\%20-\%20American\%20Dialect\%20Society.pdf}{Yod
variation in Victoria, B.C.: An acoustic-centred approach} & American
Dialect Society, Washington DC \\
~ &
\href{pdf/Thompson,\%20Onosson\%20-\%202016\%20-\%20New\%20Ways\%20of\%20Analyzing\%20Variation\%2045.pdf}{Urban
youth in Rio de Janeiro: Contemporary linguistic variation in Brazilian
Portuguese} (with Christiani Thompson) & New Ways of Analyzing Variation
45, Simon Fraser University \\
2015 &
\href{pdf/Onosson,\%20Roeder,\%20D'Arcy\%20-\%202015\%20-\%20New\%20Ways\%20of\%20Analyzing\%20Variation\%2044.pdf}{Simultaneous
Innovation \& Conservation: Unpacking Victoria's Vowels} (with Rebecca
Roeder and Alex D'Arcy) & New Ways of Analyzing Variation 44, University
of Toronto \\
~ &
\href{pdf/Rosen,\%20Onosson,\%20Li\%20-\%202015\%20-\%20New\%20Ways\%20of\%20Analyzing\%20Variation\%2044.pdf}{There's
a New Ethnolect in Town: Vowel Patterning of Filipino English in
Winnipeg} (with Nicole Rosen and Lanlan Li) & New Ways of Analyzing
Variation 44, University of Toronto \\
~ &
\href{pdf/Onosson,\%20Roeder,\%20D'Arcy\%20-\%202015\%20-\%20American\%20Dialect\%20Society.pdf}{City,
province or region: What do the vowels of Victoria tell us?} (with
Rebecca Roeder and Alex D'Arcy) & American Dialect Society, Portland
OR \\
~ &
\href{pdf/Bird,\%20Wang,\%20Onosson,\%20Benner\%20-\%202015.pdf}{Acoustic
Phonetics Lab Manual} (with Sonya Bird, Qian Wang and Allison Benner) &
Course textbook, University of Victoria \\
2014 &
\href{pdf/Onosson\%20-\%202014\%20-\%20Canadian\%20Linguistic\%20Association\%20Proceedings.pdf}{The
Prosodic Structure of Canadian Raising} & Proceedings of the Canadian
Linguistic Association \\
~ &
\href{pdf/Onosson\%20-\%202014\%20-\%20Canadian\%20Linguistic\%20Association.pdf}{The
Prosodic Structure of Canadian Raising} & Canadian Linguistic
Association, Brock University \\
~ &
\href{pdf/Onosson\%20-\%202014\%20-\%20Cascadia\%20Workshop\%20in\%20Sociolinguistics.pdf}{Analyzing
complex vowel articulations from acoustic data} & Cascadia Workshop in
Sociolinguistics, University of Victoria \\
~ &
\href{pdf/Sanderson\%20-\%202014\%20-\%20Winnipeg\%20Free\%20Press.pdf}{Finding
our voice: A primer on the Manitoba dialect} (interview) & David
Sanderson: The Winnipeg Free Press \\
2010 & \href{pdf/Onosson\%20-\%202010\%20-\%20MA\%20Thesis.pdf}{Canadian
Raising in Manitoba: Acoustic Effects of Articulatory Phasing and
Lexical Frequency} & Master's thesis, University of Manitoba \\
\bottomrule
\end{longtable}

\begin{center}\rule{0.5\linewidth}{0.5pt}\end{center}

Site built by \href{mailto:sky@onosson.com}{Sky Onosson} in
\href{https://rmarkdown.rstudio.com/index.html}{\includegraphics[width=1.13em,height=1em]{index_files/figure-latex/fa-icon-9b00320707d42527dde67262afb33ded.pdf}
Markdown}

\href{mailto:sky@onosson.com}{\includegraphics[width=1em,height=1em]{index_files/figure-latex/fa-icon-5777426a4891959e8c9a098fe5069ff6.pdf}}
\href{pdf/OnossonCV.pdf}{\includegraphics[width=0.75em,height=1em]{index_files/figure-latex/fa-icon-e71e84c2bd77dbcf0cb8f98a3ea09d57.pdf}}
\href{https://github.com/onosson}{\includegraphics[width=0.97em,height=1em]{index_files/figure-latex/fa-icon-306b3f680723aa2b0a39da2cc59a622e.pdf}}
\href{https://www.researchgate.net/profile/Sky_Onosson}{\includegraphics[width=0.88em,height=1em]{index_files/figure-latex/fa-icon-07ac794574a6b64fa79ea2d17cf66cb7.pdf}}
\href{https://www.linkedin.com/in/sky-onosson-57902870/}{\includegraphics[width=0.88em,height=1em]{index_files/figure-latex/fa-icon-c55c9b7eac4bc6a66901abe57b2a6c7d.pdf}}
\href{https://twitter.com/onosson}{\includegraphics[width=1em,height=1em]{index_files/figure-latex/fa-icon-63ba3f107f112695af1559dbf0eb8614.pdf}}

\end{document}
